\section{Acteurs du système}
Le diagramme ci-dessous illustre les **acteurs principaux** de l'application et leurs interactions avec le système.  
Chaque acteur représente un type d’utilisateur avec des droits et des responsabilités spécifiques.  

\begin{center}
\begin{tikzpicture}[actor/.style={ellipse, draw, minimum width=1.8cm, minimum height=0.8cm, align=center, font=\small}, 
                    system/.style={rectangle, draw, minimum width=3.5cm, minimum height=0.8cm, align=center, font=\small},
                    >=stealth]

% Actors
\node[actor] (visitor) {Visiteur \\ (non authentifié)};
\node[actor, below=0.6cm of visitor] (user) {Utilisateur enregistré \\ (accès complet au journal)};
\node[actor, below=0.6cm of user] (admin) {Administrateur \\ (gestion du système)};

% System box
\node[system, right=3cm of user] (systembox) {Système \\ Mental Well-being Web App};

% Connections
\draw[->] (visitor.east) -- ++(1.2,0) |- (systembox.west);
\draw[->] (user.east) -- ++(1.2,0) -- (systembox.west);
\draw[->] (admin.east) -- ++(1.2,0) |- (systembox.west);

\end{tikzpicture}

\begin{figure}[H]
\caption{Diagramme des acteurs du système}
\end{figure}
\end{center}

\noindent
Ce diagramme UML de cas d'utilisation montre les différents types d'utilisateurs (Visiteur, Utilisateur enregistré, Administrateur) et leurs interactions avec le système. 
\newline 
- Le \textit{Visiteur} peut consulter des informations publiques sans être connecté.  
\newline
- L’\textit{Utilisateur enregistré} a accès complet à son journal et aux fonctionnalités principales.  
\newline
- L’\textit{Administrateur} gère l’ensemble du système et supervise les utilisateurs et leurs données.


\section{Diagramme de cas d’utilisation}

\begin{figure}[H]
    \centering
    \includegraphics[width=0.8\textwidth]{images/usecase.png}
    \caption{Diagramme de cas d’utilisation de l’application}
\end{figure}

\section{Analyse}

Le diagramme met en évidence une séparation claire des responsabilités.
Les fonctionnalités critiques (journal, analyse, visualisation) sont réservées
aux utilisateurs authentifiés, garantissant la confidentialité des données.
