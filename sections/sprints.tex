\section{Sprints}

Le tableau ci-dessous présente l’organisation du projet selon une approche agile basée sur
des sprints successifs. Chaque sprint correspond à une phase clé du développement
et regroupe des objectifs précis. Cette organisation permet une progression
graduelle du projet, depuis la mise en place des bases techniques jusqu’à la
finalisation du MVP.

\begin{center}
\begin{table}[H]
\centering
\begin{tabular}{|c|p{10cm}|}
\hline
\textbf{Sprint} & \textbf{Objectifs / Contenu} \\
\hline
Sprint 1 & Établir les bases du système : authentification sécurisée, écriture et sauvegarde des entrées du journal, mise en place de l’architecture frontend-backend. \\ 
\hline
Sprint 2 & Introduction de l’analyse automatique des sentiments et de l’historique émotionnel : traitement des textes, génération de scores de sentiment et affichage des tendances dans l’historique utilisateur. \\ 
\hline
Sprint 3 & Finalisation du MVP : création du tableau de bord interactif, intégration des graphiques d’évolution des émotions et génération de nuages de mots pour visualiser les données de manière synthétique. \\ 
\hline
\end{tabular}
\caption{Organisation du projet en sprints et objectifs associés}
\end{table}
\end{center}
