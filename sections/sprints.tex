\section{Sprints}

Le tableau ci-dessous présente l’organisation du projet selon une approche agile basée sur
des sprints successifs. Chaque sprint correspond à une phase clé du développement
et regroupe des objectifs précis. Cette organisation permet une progression
graduelle du projet, depuis la mise en place des bases techniques jusqu’à la
finalisation du MVP.

\begin{center}
\begin{table}[H]
\centering
\begin{tabular}{|c|p{10cm}|}
\hline
\textbf{Sprint} & \textbf{Objectifs / Contenu} \\
\hline
Sprint 1 & Établir les bases du système : authentification sécurisée, écriture et sauvegarde des entrées du journal, mise en place de l’architecture frontend-backend. \\ 
\hline
Sprint 2 & Introduction de l’analyse automatique des sentiments et de l’historique émotionnel : traitement des textes, génération de scores de sentiment et affichage des tendances dans l’historique utilisateur. \\ 
\hline
Sprint 3 & Finalisation du MVP : création du tableau de bord interactif, intégration des graphiques d’évolution des émotions et génération de nuages de mots pour visualiser les données de manière synthétique. \\ 
\hline
\end{tabular}
\caption{Organisation du projet en sprints et objectifs associés}
\end{table}
\end{center}

\newpage
\section{Les interfaces}
La Figure~\ref{fig:login} illustre l’interface de connexion de l’utilisateur, assurant une authentification sécurisée et un accès contrôlé aux fonctionnalités de l’application, tout en garantissant la confidentialité des données et une expérience utilisateur ergonomique.
\begin{figure}[H]
\centering
\includegraphics[height=7cm,width=10cm]{images/login.png}
\caption{Interface de connexion de l'utilisateur}
\label{fig:login}
\end{figure}

La Figure~\ref{fig:register} présente l’interface d’inscription de l’utilisateur, permettant la création d’un compte à partir des informations essentielles, 
tout en garantissant la sécurité des données grâce au hachage des mots de passe.
\begin{figure}[H]
\centering
\includegraphics[height=8cm,width=8cm]{images/Register.png}
\caption{Interface d'inscription de l'utilisateur}
\label{fig:register}
\end{figure}

\newpage
La figure \ref{fig:acceuil} illustre l’interface d’accueil destinée à l’utilisateur. 
Cette interface constitue le premier point de contact avec l’application et présente une vue claire et intuitive des principales fonctionnalités. Elle permet à l’utilisateur de s’orienter facilement, 
d’accéder aux services essentiels et de naviguer de manière fluide grâce à une organisation ergonomique des éléments visuels. Le design vise à offrir une expérience utilisateur agréable tout en facilitant l’interaction avec le système dès la première utilisation.
\begin{figure}[H]
\centering
\includegraphics[height=7cm,width=15cm]{images/acceuil.png}
\caption{Interface d'acceuil de l'utilisateur}
\label{fig:acceuil}
\end{figure}

La figure \ref{fig:journal} illustre l’interface du journal destinée à l’utilisateur. 
Cette interface permet à l’utilisateur de rédiger et gérer une entrée de journal de façon simple, 
sécurisée et confidentielle, tout en garantissant une utilisation fluide et un contrôle total du contenu.

\begin{figure}[H]
\centering
\includegraphics[height=7cm,width=8cm]{images/journal.png}
\caption{Interface du journal de l'utilisateur}
\label{fig:journal}
\end{figure}

\newpage
La figure \ref{fig:dashboard} illustre l’interface du tableau de bord destinée à l’utilisateur. 
\newline
Ces interfaces offrent une vue d’ensemble claire et interactive des données personnelles de l’utilisateur, 
permettant une analyse approfondie de ses émotions et comportements à travers des graphiques et visualisations intuitives.
\begin{figure}[H]
\centering
\includegraphics[height=9cm,width=10cm]{images/dashboard 1.png}
\hspace{0.5cm}
\includegraphics[height=9cm,width=10cm]{images/dashboard 2.png}
\caption{Interfaces du tableau de bord de l’utilisateur}
\label{fig:dashboard}
\end{figure}
\newpage
La figure \ref{fig:history} illustre l’interface de l’historique destinée à l’utilisateur.
\newline
Cette interface offre une vue chronologique claire des anciennes entrées de journal, 
permettant à l’utilisateur de consulter ses écrits passés ainsi que les émotions associées à chaque entrée. 
Elle facilite le suivi de l’évolution émotionnelle et aide à mieux comprendre les comportements et l’état de bien-être au fil du temps.
\begin{figure}[H]
\centering
\includegraphics[height=7cm,width=10cm]{images/history.png}
\caption{Interface de l'historique de l'utilisateur}
\label{fig:history}
\end{figure}

La figure \ref{fig:insight} illustre l’interface Insights destinée à l’utilisateur.
Cette interface fournit des analyses et des visualisations intelligentes basées sur les entrées du journal, 
permettant à l’utilisateur d’identifier des tendances émotionnelles, des motifs récurrents et des indicateurs clés de son bien-être. 
Elle aide à mieux comprendre l’évolution émotionnelle et à obtenir des informations utiles pour améliorer le suivi personnel et la santé mentale.
\begin{figure}[H]
\centering
\includegraphics[height=7cm,width=10cm]{images/insight.png}
\caption{Interface de l'insight de l'utilisateur}
\label{fig:insight}
\end{figure}
