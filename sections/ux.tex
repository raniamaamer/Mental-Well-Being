L’interface utilisateur est conçue pour être à la fois minimaliste et intuitive, afin de faciliter la prise en main pour tout type d’utilisateur, même ceux ayant peu d’expérience avec les applications web. Chaque composant a été pensé pour réduire la complexité et éviter la surcharge d’informations.  

Les composants \textbf{Tailwind CSS} et \textbf{Shadcn/UI} garantissent une expérience cohérente, responsive et esthétiquement agréable sur tous les types d’appareils, que ce soit sur ordinateur, tablette ou smartphone. Grâce à leur intégration, l’interface offre :  
\begin{itemize}
    \item Des formulaires clairs et faciles à remplir pour les entrées de journal
    \item Une navigation fluide entre les différentes sections de l’application
    \item Des boutons, menus et éléments interactifs uniformes et facilement identifiables
\end{itemize}

Les indicateurs visuels, tels que les couleurs, icônes et emojis, permettent une lecture rapide et intuitive de l’état émotionnel de l’utilisateur. Par exemple :  
\begin{itemize}
    \item Des couleurs chaudes pour les émotions positives et des couleurs froides pour les émotions négatives
    \item Des emojis ou pictogrammes associés à chaque type d’émotion pour une interprétation immédiate
    \item Des graphiques simples et clairs, facilement compréhensibles, même pour un utilisateur novice
\end{itemize}

Cette approche visuelle favorise l’engagement de l’utilisateur et facilite la compréhension des tendances émotionnelles sur le long terme. De plus, l’interface est conçue pour être évolutive, permettant l’ajout futur de nouvelles fonctionnalités sans compromettre la simplicité et la clarté de l’expérience utilisateur.  

Enfin, chaque choix de design a été guidé par la volonté de créer une application accessible, agréable et efficace, combinant esthétique moderne et ergonomie optimale, ce qui contribue à la satisfaction et à la fidélisation des utilisateurs.
