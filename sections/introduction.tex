La santé mentale constitue aujourd’hui un enjeu majeur dans les sociétés modernes.
L’augmentation du stress, de l’anxiété et de la pression académique et professionnelle
impacte directement le bien-être psychologique des individus.

Avec la généralisation des technologies numériques, de nouvelles solutions émergent
pour accompagner les personnes dans la gestion de leur santé mentale. Parmi ces
solutions, les applications de journalisation personnelle occupent une place
importante, car elles permettent une expression libre des émotions et des pensées.

Le projet \textbf{Mental Well-being Web App} s’inscrit dans cette dynamique en proposant
une application web sécurisée et légère permettant aux utilisateurs de tenir un
journal personnel privé, enrichi par une analyse automatique des sentiments.
L’objectif est d’offrir un retour immédiat et visuel sur l’état émotionnel de
l’utilisateur, sans recourir à des modèles lourds ou coûteux en ressources.

Ce rapport présente une analyse complète du projet, depuis la définition des besoins
jusqu’à la conception technique, en passant par l’organisation agile, les choix
technologiques et les perspectives d’évolution.
