\section{Présentation du Product Backlog}

Le Product Backlog regroupe l’ensemble des fonctionnalités attendues du système,
classées par priorité et organisées sous forme de User Stories.  
Il sert de référence pour l’équipe de développement et permet de planifier les sprints en fonction des besoins utilisateurs et des contraintes techniques.

\section{Gestion des comptes}

Les User Stories US-1 à US-3 couvrent l’inscription, la connexion et la déconnexion,
assurant un accès sécurisé aux données personnelles.  
Ces fonctionnalités garantissent :
\begin{itemize}
    \item La création et la gestion sécurisée des comptes utilisateurs
    \item La protection des informations sensibles
    \item Une expérience utilisateur fluide et intuitive
\end{itemize}

\section{Fonctionnalités du journal}

Les User Stories US-4 à US-6 constituent le cœur fonctionnel du projet, permettant
l’écriture, la sauvegarde et la consultation des entrées.  
Ces fonctionnalités offrent :
\begin{itemize}
    \item La possibilité de noter ses expériences et émotions quotidiennes
    \item Un stockage fiable et structuré des données
    \item Un accès facile à l’historique pour suivi et analyse
\end{itemize}
\newpage
\section{Analyse intelligente}

Les User Stories US-7 et US-8 introduisent l’intelligence artificielle légère
via l’analyse automatique des sentiments et leur visualisation.  
Elles permettent :
\begin{itemize}
    \item Une compréhension rapide des tendances émotionnelles
    \item Des représentations visuelles claires grâce à des graphiques et nuages de mots
    \item Une aide à la prise de conscience personnelle pour l’utilisateur
\end{itemize}
