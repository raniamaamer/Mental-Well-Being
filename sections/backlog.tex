\section{Présentation du Product Backlog}

Le Product Backlog regroupe l’ensemble des fonctionnalités attendues du système,
classées par priorité et organisées sous forme de User Stories.  
Il sert de référence pour l’équipe de développement et permet de planifier les sprints en fonction des besoins utilisateurs et des contraintes techniques.

\section{Gestion des comptes}

Les User Stories US-1 à US-3 couvrent l’inscription, la connexion et la déconnexion,
assurant un accès sécurisé aux données personnelles.  
\newline Ces fonctionnalités garantissent :
\begin{itemize}
    \item La création et la gestion sécurisée des comptes utilisateurs
    \item La protection des informations sensibles
    \item Une expérience utilisateur fluide et intuitive
\end{itemize}

\section{Fonctionnalités du journal}

Les User Stories US-4 à US-6 constituent le cœur fonctionnel du projet, permettant
l’écriture, la sauvegarde et la consultation des entrées.  
\newline Ces fonctionnalités offrent :
\begin{itemize}
    \item La possibilité de noter ses expériences et émotions quotidiennes
    \item Un stockage fiable et structuré des données
    \item Un accès facile à l’historique pour suivi et analyse
\end{itemize}
\newpage
\section{Analyse intelligente}

Les User Stories US-7 et US-8 introduisent l’intelligence artificielle légère
via l’analyse automatique des sentiments et leur visualisation.  
Elles permettent :
\begin{itemize}
    \item Une compréhension rapide des tendances émotionnelles
    \item Des représentations visuelles claires grâce à des graphiques et nuages de mots
    \item Une aide à la prise de conscience personnelle pour l’utilisateur
\end{itemize}
\section{Product Backlog}

Le Product Backlog présente de manière synthétique l’ensemble des User Stories
du projet, classées par priorité et organisées selon les grands thèmes
fonctionnels. Il constitue un outil central de pilotage du projet agile et
permet d’assurer une vision globale des fonctionnalités à développer.
\begin{center}
\begin{table}[H]
\centering
\begin{tabular}{|c|c|p{9cm}|}
\hline
\textbf{ID} & \textbf{Priorité} & \textbf{Description de la User Story} \\
\hline
US-1 & Élevée & Inscription sécurisée des utilisateurs avec chiffrement des mots de passe. \\
\hline
US-2 & Élevée & Connexion sécurisée permettant l’accès au tableau de bord personnel. \\
\hline
US-3 & Moyenne & Déconnexion de l’utilisateur afin de garantir la sécurité des données. \\
\hline
US-4 & Élevée & Mise à disposition d’un éditeur de texte pour l’écriture du journal. \\
\hline
US-5 & Élevée & Sauvegarde des entrées du journal avec horodatage. \\
\hline
US-6 & Élevée & Consultation chronologique des entrées précédentes. \\
\hline
US-7 & Élevée & Analyse automatique du sentiment des textes enregistrés. \\
\hline
US-8 & Moyenne & Affichage d’indicateurs visuels représentant le sentiment associé. \\
\hline
US-9 & Élevée & Visualisation de l’évolution émotionnelle via un graphique linéaire. \\
\hline
US-10 & Moyenne & Génération d’un nuage de mots à partir des entrées du journal. \\
\hline
\end{tabular}
\caption{Product Backlog du projet Mental Well-being Web App}
\end{table}
\end{center}