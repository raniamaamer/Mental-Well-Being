\section{Vision du projet}

La vision du projet \textbf{Mise en place d'une Application Web d'analyse de la Santé Mentale} est de concevoir une application
web simple, intuitive et sécurisée permettant à chaque utilisateur de suivre son
bien-être mental à travers l’écriture quotidienne.

Contrairement aux applications complexes ou médicalisées, cette solution se veut
accessible à tous, centrée sur l’auto-réflexion et la prise de conscience émotionnelle.

\section{Objectifs}

Les objectifs principaux du projet sont :
\begin{itemize}
    \item Offrir un espace d’écriture personnel et privé
    \item Analyser automatiquement le sentiment des textes rédigés
    \item Visualiser l’évolution émotionnelle dans le temps
    \item Garantir la confidentialité et la sécurité des données
\end{itemize}

\section{Besoins fonctionnels}

Les besoins fonctionnels couvrent :
\begin{itemize}
    \item Gestion des comptes utilisateurs
    \item Journalisation des entrées
    \item Analyse de sentiment automatisée
    \item Tableau de bord avec visualisations
    \item Fonctionnalités avancées (export, rappels – futures)
\end{itemize}

\section{Besoins non fonctionnels}

Les besoins non fonctionnels définissent les exigences de qualité et de performance de l’application afin d’assurer son bon fonctionnement.

\begin{itemize}
    \item \textbf{Sécurité et confidentialité des données} :  
    L’application doit garantir la protection des données des utilisateurs grâce à des mécanismes de sécurité et de contrôle d’accès appropriés.

    \item \textbf{Performance et réactivité} :  
    Le système doit offrir des temps de réponse rapides et une utilisation fluide.

    \item \textbf{Accessibilité} :  
    L’application doit être accessible sur différents appareils avec une interface adaptée à chaque écran.

    \item \textbf{Scalabilité} :  
    Le système doit supporter l’augmentation du nombre d’utilisateurs sans perte de performance.
\end{itemize}


