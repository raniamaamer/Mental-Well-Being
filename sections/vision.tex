\section{Vision du projet}

La vision du projet \textbf{Mise en place d'une Application Web d'analyse de la Santé Mentale} est de concevoir une application
web simple, intuitive et sécurisée permettant à chaque utilisateur de suivre son
bien-être mental à travers l’écriture quotidienne.

Contrairement aux applications complexes ou médicalisées, cette solution se veut
accessible à tous, centrée sur l’auto-réflexion et la prise de conscience émotionnelle.

\section{Objectifs}

Les objectifs principaux du projet sont :
\begin{itemize}
    \item Offrir un espace d’écriture personnel et privé
    \item Analyser automatiquement le sentiment des textes rédigés
    \item Visualiser l’évolution émotionnelle dans le temps
    \item Garantir la confidentialité et la sécurité des données
\end{itemize}

\section{Besoins fonctionnels}

Les besoins fonctionnels couvrent :
\begin{itemize}
    \item Gestion des comptes utilisateurs
    \item Journalisation des entrées
    \item Analyse de sentiment automatisée
    \item Tableau de bord avec visualisations
    \item Fonctionnalités avancées (export, rappels – futures)
\end{itemize}

\section{Besoins non fonctionnels}
Les besoins non fonctionnels incluent :
\begin{itemize}
    \item Sécurité et confidentialité des données
    \item Performance et réactivité de l’application
    \item Accessibilité sur divers appareils
    \item Scalabilité pour gérer la croissance des utilisateurs
\end{itemize}
