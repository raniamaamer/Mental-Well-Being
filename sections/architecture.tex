\section{Architecture générale} 
Le projet repose sur une architecture client–serveur séparant clairement le frontend et le backend. Cette organisation améliore la maintenabilité, l’évolutivité et la clarté du système. Le frontend gère l’interface utilisateur, tandis que le backend assure le traitement des données, la sécurité, la persistance et 
l’intelligence artificielle, avec une communication basée sur des API REST au format JSON.

\section{Frontend}
\begin{center}
    \includegraphics[height=2cm]{images/nextjs.png} \\
    \textbf{Next.js 14}
\end{center}

Le frontend est développé avec \textbf{Next.js 14}, un framework basé sur React offrant de hautes performances grâce à la combinaison du rendu côté serveur et côté client. Il permet une structuration claire des composants, 
une navigation fluide entre les différentes pages et une communication efficace avec les API du backend Flask, 
améliorant ainsi l’expérience utilisateur et les temps de chargement.

\textbf{Next.js} permet notamment:
\begin{itemize}
    \item Une interface réactive
    \item Un rendu performant
    \item Une navigation fluide
\end{itemize}
L’utilisation de ce framework facilite également l’intégration de bibliothèques
tierces telles que Chart.js et contribue à une expérience utilisateur moderne,
responsive et cohérente sur différents types d’appareils.



\section{Backend}
\begin{center}
    \includegraphics[height=2cm]{images/flask.png} \\
    \textbf{Flask}
\end{center}

Le backend est développé avec \textbf{Flask}, un micro-framework Python léger et flexible, adapté à la création d’API REST. Il centralise la logique métier du système, notamment l’authentification, 
la gestion des sessions, le stockage des données et l’analyse de sentiment, 
tout en assurant une communication sécurisée et structurée avec le frontend.

Le backend \textbf{Flask} prend en charge:
\begin{itemize}
    \item La gestion des utilisateurs
    \item La gestion des entrées de journal
    \item L’analyse de sentiment
\end{itemize}

Ce choix technologique permet également une intégration naturelle avec les
bibliothèques de traitement du langage naturel (TextBlob, VADER, spaCy),
renforçant ainsi la dimension intelligente de l’application.


\section{Visualisation}

\begin{center}
    \includegraphics[height=3cm]{images/chartjs.png} \\
    \textbf{Chart.js}
\end{center}

La visualisation des données dans le projet est assurée par \textbf{Chart.js}, une bibliothèque JavaScript qui permet de créer des graphiques interactifs et dynamiques.  

Grâce à \textbf{Chart.js}, le système peut afficher:
\begin{itemize}
    \item Des graphiques d’évolution des émotions au fil du temps
    \item Des tableaux de bord synthétiques
    \item Des nuages de mots basés sur les entrées du journal
\end{itemize}
Cette visualisation permet à l’utilisateur de mieux comprendre ses tendances émotionnelles et d’analyser ses données de manière intuitive et interactive.

\section{Base de données}
\begin{center}
    \includegraphics[height=2cm]{images/sqlite.png} \\
    \textbf{SQLite}
\end{center}

La base de données utilisée est \textbf{SQLite}, une base de données relationnelle
légère intégrée directement à l’application. Ce choix est particulièrement
adapté à un prototype ou à un MVP, car il ne nécessite pas de serveur de base de
données dédié et offre une grande simplicité de déploiement.

\textbf{SQLite} permet de stocker de manière structurée et sécurisée les données
essentielles de l’application tout en garantissant de bonnes performances pour
un volume modéré de données.

La base de données \textbf{SQLite} stocke:
\begin{itemize}
    \item Les utilisateurs
    \item Les entrées de journal
    \item Les scores de sentiment
\end{itemize}

Ce choix constitue une base solide pour le projet, avec la possibilité
d’une migration future vers une solution plus robuste (comme PostgreSQL) si le
volume d’utilisateurs venait à augmenter.
