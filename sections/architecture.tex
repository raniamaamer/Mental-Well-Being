\section{Architecture générale}
Le projet adopte une architecture client–serveur avec séparation claire entre
le frontend et le backend.

\section{Frontend}
\begin{center}
    \includegraphics[height=2cm]{images/nextjs.png} \\
    \textbf{Next.js 14}
\end{center}
Le frontend est développé avec \textbf{Next.js 14}, permettant :
\begin{itemize}
    \item Une interface réactive
    \item Un rendu performant
    \item Une navigation fluide
\end{itemize}

\section{Backend}
\begin{center}
    \includegraphics[height=2cm]{images/flask.png} \\
    \textbf{Flask}
\end{center}
Le backend repose sur \textbf{Flask}, exposant des API REST sécurisées pour :
\begin{itemize}
    \item La gestion des utilisateurs
    \item La gestion des entrées de journal
    \item L’analyse de sentiment
\end{itemize}

\section{Visualisation}

\begin{center}
    \includegraphics[height=3cm]{images/chartjs.png} \\
    \textbf{Chart.js}
\end{center}

La visualisation des données dans le projet est assurée par \textbf{Chart.js}, une bibliothèque JavaScript qui permet de créer des graphiques interactifs et dynamiques.  

Grâce à Chart.js, le système peut afficher :
\begin{itemize}
    \item Des graphiques d’évolution des émotions au fil du temps
    \item Des tableaux de bord synthétiques
    \item Des nuages de mots basés sur les entrées du journal
\end{itemize}
Cette visualisation permet à l’utilisateur de mieux comprendre ses tendances émotionnelles et d’analyser ses données de manière intuitive et interactive.

\section{Base de données}
\begin{center}
    \includegraphics[height=2cm]{images/sqlite.png} \\
    \textbf{SQLite}
\end{center}
La base de données SQLite stocke :
\begin{itemize}
    \item Les utilisateurs
    \item Les entrées de journal
    \item Les scores de sentiment
\end{itemize}
