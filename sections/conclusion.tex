Le projet \textbf{Mental Well-being Web App} répond pleinement aux objectifs définis dans le cahier des charges. Il propose une solution fonctionnelle, sécurisée et intelligente pour le suivi du bien-être mental, combinant une interface utilisateur intuitive avec des fonctionnalités d’analyse émotionnelle fiables.  

Ce travail constitue une base solide pour des évolutions futures et illustre la pertinence de l’utilisation d’une intelligence artificielle légère dans des applications à impact humain. Il démontre également l’efficacité d’une architecture web moderne, intégrant des technologies comme \textbf{Next.js}, \textbf{Flask} et \textbf{Chart.js}, pour offrir une expérience fluide et engageante.  

Les perspectives d’évolution comprennent :  
\begin{itemize}
    \item \textbf{Migration vers PostgreSQL} pour améliorer la scalabilité et la robustesse de la base de données.  
    \item \textbf{Utilisation de modèles NLP avancés} afin d’obtenir une analyse plus fine et contextuelle des sentiments.  
    \item \textbf{Développement d’une application mobile} pour offrir un suivi du bien-être accessible à tout moment.  
    \item \textbf{Notifications intelligentes} pour alerter et accompagner l’utilisateur de manière personnalisée selon son état émotionnel.  
\end{itemize}

Dans l’ensemble, le projet illustre la capacité d’un système léger et bien conçu à répondre aux besoins des utilisateurs tout en restant évolutif et performant.
