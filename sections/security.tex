La sécurité constitue un axe fondamental du projet, garantissant la protection des données personnelles et la confiance des utilisateurs.  

\textbf{Chiffrement des mots de passe :} Tous les mots de passe sont sécurisés avec l’algorithme \textbf{bcrypt} avant stockage. Bcrypt offre une grande résistance aux attaques par force brute grâce au salage et à un temps de calcul ajustable, garantissant que les mots de passe restent illisibles même en cas de fuite de la base de données.  

\textbf{Gestion des sessions :} Les sessions utilisateurs sont gérées via \textbf{Flask-Login}, assurant une authentification fiable et la limitation de l’accès aux données sensibles. Chaque session est unique et expire automatiquement après une période d’inactivité pour prévenir tout accès non autorisé.  

\textbf{Chiffrement des données sensibles :} Une évolution future prévoit l’intégration d’un chiffrement \textbf{AES (Advanced Encryption Standard)} pour protéger les informations critiques et renforcer la confidentialité des données stockées.  

\textbf{Bonnes pratiques :} Des mesures complémentaires sont également mises en place, telles que :
\begin{itemize}
    \item Validation des entrées utilisateurs pour éviter les injections SQL et attaques XSS.
    \item Protection des formulaires contre les attaques CSRF.
    \item Limitation des tentatives de connexion pour prévenir les intrusions.
    \item Journalisation des accès et actions critiques pour détecter tout comportement suspect.
\end{itemize}

\textbf{Bénéfices pour l’utilisateur :} Cette approche multi-niveaux permet aux utilisateurs de saisir leurs informations personnelles et émotionnelles en toute confiance, avec une expérience à la fois simple, fluide et sécurisée.  

\textbf{Résumé :} En combinant chiffrement des mots de passe, gestion sécurisée des sessions et protections avancées, le projet place la sécurité au cœur de sa conception, garantissant ainsi une utilisation fiable et protégée.
